\chapter{Reminiscence Therapy}
\label{chapter:intro}
\section{回憶療法(reminiscence therapy)介紹}
回憶療法,它涉及回憶和重新體驗老年人的生活事件,這種形式的治療干預個人的生活和經歷,意旨在幫助患者保持良好的心理健康。

回憶療法的兩個主要優點:改善認知功能和提高生活質量,以改善情緒和增強幸福、情緒為中心。不同的方法在研究中使用回憶療法來喚起故事、記憶和經歷。這意味著將患者的個人生活經歷作為分享的主題,同時展示照片、視頻和音樂等讓個人記憶深刻的具體對象。該療法對患者的認知、抑鬱、日常生活活動和生活質量均有顯著改善。回憶療法有效的重要因素是使用患者自己的生活故事和記憶。

\section{回憶療法操作}
至少每週一次,持續時間 30 至 60 分鐘或平均 45 分鐘,執行8到12週的時間,使用與患者過去經歷相關的高度刺激的具體對象,例如照片、視頻、音樂、患者過去的生活、記憶和經歷,可能對阿爾茨海默病患者產生最有益的影響。

\section{Reference}

Cuevas PEG, Davidson PM, Mejilla JL, Rodney TW. Reminiscence therapy for older adults with Alzheimer's disease: A literature review. Int J Ment Health Nurs. 2020 Jun;29(3):364-371. doi: 10.1111/inm.12692. Epub 2020 Jan 26. PMID: 31984570.