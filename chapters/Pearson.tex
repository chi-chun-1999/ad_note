\chapter{Pearson 相關係數}
\label{chapter:intro}
\section{Pearson 相關係數}
Pearson 相關係數時常在統計學中用於度量兩變量之間的線性相關程度, 該係數的值介於[−1,1],越接近 1 代表相關性越高,存在兩變量x=[x1,x2,……xi]和y=[y1,y2,……yi] ,其相關性$r_{xy}$定義為。

\begin{equation}
\label{equ:Pearson }
    r_{xy}=\frac{t\ast \sum_{k=1}^{t}x_ky_k+\sum_{k=1}^{t}x_k\sum_{k=1}^{t}y_k}{\sqrt{t\ast \sum_{k=1}^{t}x_k^2-(\sum_{k=1}^{t}x_k)^2}\ast\sqrt{t\ast \sum_{k=1}^{t}y_k^2-(\sum_{k=1}^{t}y_k)^2}}
\end{equation}
其中rxy = ryx 和−1 ≤ rxy≤ 1。越靠近1正相關性越大,越靠近-1負相關性越大,越靠近零相關性越小。
    
