\chapter{Pearson product-moment correlation coefficient}
\label{chapter:intro}
皮爾森積動差相關係數(Pearson product-moment correlation coefficient,PPMCC),時常在統計學中用於度量兩變量之間的線性相關程度, 該係數的值介於[−1,1],越接近 1 代表相關性越高,存在兩變量 \(x=[x_1,x_2,...,x_i]\) 和 \(y=[y_1,y_2,...,yi]\)  ,其相關性 \(r_{xy}\) 定義如下式:

%%\begin{equation}
%%\label{eqn:Pearson }
%%    r_{xy}=\frac{t\ast \sum_{k=1}^{t}x_ky_k+\sum_{k=1}^{t}x_k\sum_{k=1}^{t}y_k}{\sqrt{t\ast \sum_{k=1}^{t}x_k^2-(\sum_{k=1}^{t}x_k)^2}\ast\sqrt{t\ast \sum_{k=1}^{t}y_k^2-(\sum_{k=1}^{t}y_k)^2}}
%%\end{equation}

$$r_{xy}=\frac{t\ast \sum_{k=1}^{t}x_ky_k+\sum_{k=1}^{t}x_k\sum_{k=1}^{t}y_k}{\sqrt{t\ast \sum_{k=1}^{t}x_k^2-(\sum_{k=1}^{t}x_k)^2}\ast\sqrt{t\ast \sum_{k=1}^{t}y_k^2-(\sum_{k=1}^{t}y_k)^2}}$$


其中 \(r_{xy}\)  範圍1跟-1之間,越靠近1正相關性越大,越靠近-1負相關性越大,越靠近零相關性越小。
    
