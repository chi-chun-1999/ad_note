\chapter{Pearson product-moment correlation coefficient}
\label{chapter:intro}
皮爾森積動差相關係數(Pearson product-moment correlation coefficient,PPMCC)時常在統計學中用於度量兩變量之間的線性相關程度, 該係數的值介於[−1,1],越接近 1 代表相關性越高,存在兩變量x=[$x_1,x_2,……x_i$]和y=[$y_1$,$y_2$,……$y_i$] ,t為總數量,其相關性$r_{xy}$定義為。

\begin{equation}
\label{eqn:Pearson }
    r_{xy}=\frac{t\times\sum_{k=1}^{t}x_ky_k+\sum_{k=1}^{t}x_k\sum_{k=1}^{t}y_k}{\sqrt{t\times \sum_{k=1}^{t}x_k^2-(\sum_{k=1}^{t}x_k)^2}\times\sqrt{t\times\sum_{k=1}^{t}y_k^2-(\sum_{k=1}^{t}y_k)^2}}
\end{equation}
其中$r_{xy}$ 範圍1跟-1之間,越靠近1正相關性越大,越靠近-1負相關性越大,越靠近零相關性越小。
    