\chapter{失智症之認知功能評估}
\label{chapter:intro}
\section{簡易智能評估(Mini-Mental State Examination,MMSE)}
為臨床上最廣泛應用評估老年人認知狀態之工具,共分為定向感、記錄能力、注意力和計算力、回憶能力、抽象概念和語言能力來評估,滿分是三十分,答對一項給一分,若低於二十三分表度知能損傷,低於十六分表重度認知功能損傷。
是Folstein等人1975年所提出,2001年3月MMSE版權已轉為PAR, Inc.所有,任何臨床上使用、論文發表及研究使用需至PAR,Inc.網站申請版權使用。
認知評估項目:
\begin{enumerate}
	\item
    定向感(時間地點)
	\item
	注意力與算術能力
	\item
    立即記憶與短期記憶
	\item
	語言能力:讀、寫、命名、理解與操作
	\item
    視覺空間能力
\end{enumerate}

\label{sec:background}
\section{臨床失智評估表 ( Clinical Dementia Rating,CDR)}

具有輕微認知症狀(例如記憶力下降)的患者是否患有前期或臨床前阿茲海默氏病(Alzheimer disease,AD)並在不久的將來發展為 AD 癡呆仍然是臨床醫生面臨的挑戰。這項任務對於正確和早期的 AD 診斷、開始對AD症治療、規劃未來以及希望很快開始改善疾病的治療都至關重要。

美國華盛頓大學的Hughes等人提出,用來評估阿茲海默 症患者日常生活及認知功能作整體性評估的量表。

受試者部份:注意力、長期記憶、近期記憶、時空、定向感、語言、抽象與判斷等作評估。

家屬部份:詢問家人或主要照顧者,受試者在日常生活是否出現與記憶力有關的問題或困擾。若有狀況,再進一步詢問症狀何時顯露?發展速度有多快?症狀進展如何?



\section{神經心理衡鑑(CASI)}
認知功能障礙篩檢量表 (CASI):包含25題,滿分為一百分,經由固定的換算方式可以把它化成9個認知功能的細項,分別是長期記憶,短期記憶、時間空間定向感、注意力、心智操作和集中力 、思考流暢度,語言及基本認知功能、抽象思考能力及判斷力和手眼協調構圖能力。
